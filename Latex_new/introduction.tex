\documentclass[a4paper, UKenglish, 11pt]{uiomaster}
\usepackage{lipsum}
\usepackage[subpreambles=true]{standalone}


\begin{document}
\chapter{Introduction}
Electroencephalography (EEG) is a method for recording electric potentials stemming from neural activity at the surface of the human head, and it has important scientific and clinical applications. An important issue in EEG signal analysis is so-called source localization where the goal is to localize the source generators, that is, the neural populations that are generating specific EEG signal components. An important example is the localization of the seizure onset zone in EEG recordings from patients with epilepsy. A drawback of EEG signals is however that they tend to be difficult to link to the exact neural activity that is generating the signals.

Source localization from EEG signals has been extensively investigated during the last decades, and a large variety of different methods have been developed. Source localization is very technically challenging: because the number of EEG electrodes is far lower than the number of neural populations that can potentially be contributing to the EEG signal, the problem is mathematically under-constrained, and additional constraints on the number of neural populations and their locations must therefore be introduced to obtain a unique solution. 

For the purpose of analyzing EEG signals, the neural sources are treated as equivalent current dipoles. This is because the electric potentials stemming from the neural activity of a population of neurons will tend to look like the potential from a current dipole when recorded at a sufficiently large distance, as in EEG recordings. Source localization is therefore typically considered completed when the location of the current dipoles has been obtained. However, an exciting possibility is to try to go one step further and identify the type of neural activity that caused a localized current dipole. For example, the type of synaptic input (excitatory or inhibitory) to a population of neurons, and the location of the synaptic input (apical or basal) will result in different current dipoles (Ness et al., 2022). It has also been speculated that dendritic calcium spikes can be detected from EEG signals, which could lead to exciting new possibilities for studying learning mechanisms in the human brain (Suzuki $\&$ Larkum, 2017). Identifying different types of neural activity from EEG signals would however require knowledge of how different types of neural activity are reflected in EEG signals. Tools for calculating EEG signals from biophysically detailed neural simulations have however recently been developed, and are available through the software LFPy 2.0 (Hagen et al., 2018; Næss et al., 2021). This allows for simulations of different types of neural activity and the resulting EEG signals, opening up for a more thorough investigation of the link between EEG signals and the underlying neural activity.

The past decade has seen a rapid increase in the availability and sophistication of machine learning techniques based on artificial neural networks, like Convolutional Neural Networks (CNNs). These methods have also been applied to EEG source localization with promising results. However, it has not been investigated if CNNs can also identify the neural origin of EEG signals, in addition to localizing neural sources. In this Master’s thesis, the aim will be to investigate the possibility of using CNNs to not only localize current dipoles but also identify the neural origin of different types of neural activity, based on simulated data of different types of neural activity and the ensuing EEG signal.
\end{document}