% !TEX root = main.tex

\documentclass[a4paper, UKenglish, 11pt]{uiomaster}
\usepackage{lipsum}
\usepackage[subpreambles=true]{standalone}

\begin{document}

\chapter{Basic Concepts within Neuroscience} \label{chap:intro_neuro}

Neuroscience is a multidisciplinary field focused on understanding the complexities of the human brain and nervous system. At its core, neuronal communication forms the foundation for brain function, where billions of neurons interact through electrical signals. Electroencephalography (EEG) plays a pivotal role in recording and analyzing the electrical communication in the brain. Most importantly EEG serves as a non-invasive tool to detect abnormal brain activity and identify neurological disorders such as epilepsy.

In this introductory chapter, we will explore some of the aspects of neuronal communication, which will serve as a foundation for Chapter 2, where we will delve into the principles of EEG recordings. By familiarizing ourselves with some of the basic concepts within neuroscience, we aim to gain deeper insights into the applications of EEG within this dynamic field. The chapter is based on the books \textit{Neuronal Dynamics} by Gerstner, Kistler, Naud, and Paninski \cite{gerstner2014neuronal} and \textit{Principles of Computational Modelling in Neuroscience} by Sterratt, Graham, Gillies, and Willshaw \cite{sterratt2011principles}.

% Exitatory and inhibatory !!

% NEW IDEAS:
% Final section: abnormal electrical signals, recordings, eeg --- will introduce in next chapter

\section{The Neuron}
Neurons are the fundamental units of the central nervous system, forming intricate networks with numerous interconnections. Similar to other cells, neurons have a voltage difference across their cell membrane known as the membrane potential. This potential is a result of the selective permeability of the cell membrane to different ions, particularly sodium Na$^+$, calcium Ca2$^+$, and chloride Cl$^-$. At rest, the neuron maintains a relatively higher concentration of sodium ions outside the cell and a higher concentration of potassium ions inside the cell. This difference in ion concentrations, along with the presence of ion channels that regulate the flow of ions in and out of the cell, contributes to the resting membrane potential. Typically, the membrane potential of a neuron is around $-$65 mV, indicating that the interior of the cell is negatively charged compared to the external environment \cite{sterratt2011principles}.

A neuron consists of three distinct parts: the \emph{dendrites}, the \emph{soma}, and the \emph{axon}. Dendrites, with their branching structure, play an important role in collecting signals from other neurons. These signals are transmitted to the soma, which acts as the central processing unit, performing essential nonlinear transformations of signals. If the total input received by the soma reaches a specific threshold, an \emph{action potential} is initiated. This signal generates an electrical current that travels along the axon. When the electrical current reaches the \emph{synaptic cleft}, chemical messengers known as \emph{neurotransmitters} are released. These messengers play the key role in transmitting the signal to the next neuron. If \emph{receptors} on the receiving neuron accept the neurotransmitters, a new electrical signal is generated \cite{gerstner2014neuronal}. A basic illustration of a single neuron is presented in Figure \ref{fig:neuron}.

\begin{figure}
    \centering
    \includegraphics[width=1.0\linewidth]{figures/Neuron_wikimedia.png}
    \caption{Illustation of single neuron with dendrites, soma and axon. The figure has been adapted from Wikimedia Commons with attribution to Quasar Jarosz at English Wikipedia and is licensed under the Creative Commons Attribution-Share Alike 3.0 Unported license \cite{wikimedia-neuron}. The figure has been edited to remove some of the original details.}
    \label{fig:neuron}
\end{figure}


% \subsection{Some title leading to EEG abnormal shape }
When the electrical signals transmitted towards the soma reach a specific threshold value, typically around -55 mV, the neuron initiates an action potential -- and we say that the neuron \emph{fires}. This initiation of an action potential, when observed in intracellular recordings, can be seen as a spike with an amplitude of about 100 mV and a duration of 1-2 ms \cite{gerstner2014neuronal}. In Figure \ref{fig:action_potential}, we have provided an illustration of the typical intracellular action potential.

\begin{figure}
    \centering
    \includegraphics[width=0.8\linewidth]{figures/action_potential.png}
    \caption{Illustation of a characteristic action potential. The figure have been adapted from Wikimedia Commons and are licensed under the Creative Commons Attribution-Share Alike 3.0 Unported license \cite{wikimedia-action}. The figure has been edited to remove some of the original details.}
    \label{fig:action_potential}
\end{figure}

The action potential is characterized by a rapid increase in electrical potential, marked by a sharp, positive spike, followed by a quick return to the resting state. This waveform retains a consistent shape as it propagates along the axon. \rednote{Is this right formulation?} Therefore, when investigating spike patterns, the emphasis is not typically on the waveform's shape, but rather, in the analysis of the number and timing of action potentials emitted by the neuron, which are commonly referred to as \emph{spike trains}.


Spike trains observed during epileptic seizures, which can be seen in EEG recordings, exhibit distinguishable characteristics compared to spike trains in regular neural activity. Epileptic seizures result from bilateral synchronous firing of neurons and manifest as specific EEG patterns known as \emph{spike-and-wave discharges}. These discharges exhibit repetitive and rhythmic patterns, typically occurring at a frequency of around 2.5 Hz, setting them apart from the irregular and asynchronous firing of action potentials seen in typical spike trains\cite{gerstner2014neuronal}. During epileptic events, the initiation of spike-and-wave discharges involves complex mechanisms, including the interplay of voltage-gated sodium and calcium channels and the role of inhibitory postsynaptic potentials \cite{wiki:electroencephalography}. \rednote{Hvis sant: Vet du om en kilde til dette Torbjørn?.}
%This stark contrast in the rhythmicity and temporal dynamics of epileptic spike trains highlights the distinct nature of epileptic activity when compared to the more variable and less periodic patterns observed in "normal" neuronal firing \cite{gerstner2014neuronal}.

% Need to be rewritten somehow.
% The hypersynchronous discharges that occur during a seizure may begin in a very discrete region of cortex and then spread to neighboring regions. Seizure initiation is characterized by two concurrent events: 1) high-frequency bursts of action potentials, and 2) hypersynchronization of a neuronal population. The synchronized bursts from a sufficient number of neurons result in a so-called spike discharge on the EEG. At the level of single neurons, epileptiform activity consists of sustained neuronal depolarization resulting in a burst of action potentials, a plateau-like depolarization associated with completion of the action potential burst, and then a rapid repolarization followed by hyperpolarization. This sequence is called the paroxysmal depolarizing shift. (Slide 23) The bursting activity resulting from the relatively prolonged depolarization of the neuronal membrane is due to influx of extracellular Ca++, which leads to the opening of voltage-dependent Na+ channels, influx of Na+, and generation of repetitive action potentials. The subsequent hyperpolarizing afterpotential is mediated by GABA receptors and Cl− influx, or by K+ efflux, depending on the cell type.

% spike-and-vale / epileptiform activity  ... can be measured on eeg recordings... blabla .... spike trains looks like for epilepsy ... can be picked up on eeg recordings
% MAYBE NOT INCLUDE anything about eeg here, if anything, include how one can pick up on abnormal activity  ??

\section{The Cerebral Cortex}
The brain is typically divided into various regions, with the \emph{cortex} characterized as a thin and folded sheet of neurons. Different cortical areas have specific roles. \cite{gerstner2014neuronal} Within the oldest part of the cortex, we find the hippocampus, which plays a major role in learning and memory functions. Within this component, the development of some common epilepsy syndromes tends to arise, making this part of the brain a topic of particular interest \cite{bromfield2006introduction}.

The cerebral cortex is a folded structure with varying thickness, typically ranging from 2 to 5 mm and covering an approximate surface area of 1600 to 4000 cm$^2$ \cite{nunez2006electric}. Within the cortex, billions of neurons exist, forming strong interconnections -- where single presynaptic neurons can connect to more than 10,000 postsynaptic neurons. While many axonal branches end close to the neuron itself, some axons extend several centimeters to reach neurons in other brain regions \cite{gerstner2014neuronal}.

There are typically two primary classes of neurons in the cerebral cortex. \emph{Pyramidal neurons} send information to distant areas of the brain, playing a crucial role in long-distance communication. On the other hand, \emph{interneurons} are considered local-circuit neurons, exerting their influence on nearby neurons. Most pyramidal neurons form excitatory synapses, stimulating post-synaptic neurons, while most interneurons form inhibitory synapses, suppressing the activity of pyramidal cells or other inhibitory neurons \cite{bromfield2006introduction}. Within EEG recordings, pyramidal cells within the cortex are recognized as the primary source of the electrical potentials measured.



% \rednote{Fill in from sources, paper etc. https://www.cell.com/current-biology/pdf/S0960-9822(11)01198-5.pdf, https://www.ncbi.nlm.nih.gov/books/NBK2510/}

% \section{Head Models and Multicompartmental Modeling }
% % Something more about volume conducters
%
% \section{Currents and Potentials in the Brain}
% Ohm's law in volume conductors is a more genral statement than its usual form in electrical circuits. It is a linear relationship between vector current density $J$ and the electric field $E$. The law is then expessed as follows:
%
% \begin{equation}
% J = \sigma E,
% \label{eq:ohms_law}
% \end{equation}
%
% where $\sigma$ is the conductivity of the .... (physical material). (Soruce: Electric Fields of the Brain: The Neurophysics of EEG).


% \rednote{rewrite}
% When a single pyramidal cell is stimulated and reaches its threshold, it generates an action potential. During this process, the synapse receives an excitatory signal, leading to a post-synaptic potential where positively charged ions enter the cell. As a result, a relatively negative charge is induced in the nearby extracellular space, which refers to the fluid-filled space surrounding the neuron. As the electrical signal travels down the dendrite, it eventually exits the cell membrane at locations further away from the synapse, and these locations are referred to as the "source." Consequently, an outward flow of positive charge prevails, leading to a relatively positive charge in the extracellular space. This spatial configuration creates an external dipole, with a relatively negative charge at the distant part of the dendrite and a positive charge closer to the cell body \cite{bromfield2006introduction}.


\end{document}
