\documentclass[a4paper, UKenglish, 11pt]{uiomaster}
\usepackage{lipsum}
\usepackage[subpreambles=true]{standalone}


\begin{document}

% TODO:
\section{Introduction}
Electroencephalography (EEG) is a method for recording electric potentials stemming from neural activity at the surface of the human head, and it has important scientific and clinical applications. An important issue in EEG signal analysis is the so-called EEG inverse problem where the goal is to localize the source generators, that is, the neural populations that are generating specific EEG signal components. An important example is the localization of the seizure onset zone in EEG recordings from patients with epilepsy. A drawback of EEG signals is however that they tend to be difficult to link to the exact neural activity that is generating the signals.

Source localization from EEG signals has been extensively investigated during the last decades, and a large variety of different methods have been developed. Source localization is very technically challenging: because the number of EEG electrodes is far lower than the number of neural populations that can potentially be contributing to the EEG signal, the problem is mathematically under-constrained. Additional constraints on the number of neural populations and their locations must therefore be introduced to obtain a unique solution.

For the purpose of analyzing EEG signals, the neural sources are treated as equivalent current dipoles. This is because the electric potentials stemming from the neural activity of a population of neurons will tend to look like the potential from a current dipole when recorded at a sufficiently large distance, as in EEG recordings. Source localization is therefore typically considered completed when the locations of the current dipoles have been obtained. Tools for calculating EEG signals from biophysically detailed neural simulations have however recently been developed, and are available through the software LFPy 2.0 \cite{LFPy}. This allows for simulations of different types of neural activity and the resulting EEG signals, opening up for a more thorough investigation of the link between EEG signals and the underlying neural activity.

The past decade has seen a rapid increase in the availability and sophistication of machine learning techniques based on artificial neural networks (ANNs). These methods have also been applied to EEG source localization with promising results. Prior research has demonstrated ANNs achieving localization errors below 5$\%$, exhibiting computational efficiency, and robustness against measurement noise compared to traditional methods \cite{van2000eeg}. In 2019 Tankelevich introduced a deep feed-forward network capable of discerning the correct source clusters within scalp signals, representing a notable advancement in distributed dipole solutions \cite{tankelevich2019inverse}. Moreover, in a recent study, the authors embarked on an exploration of Convolutional Neural Networks (CNNs) to tackle the EEG inverse problem. This network was designed to identify multiple sources using training data adhering to biologically plausible constraints \cite{hecker2021convdip}.

In this master's thesis, the primary objective is to investigate the feasibility of employing both a fully-connected neural network and a convolutional neural network for localizing current dipoles. This investigation extends to identifying the strength and radius of dipole clusters, relying on self-simulated EEG data generated with tools available through the LFPy software.


% Electroencephalography (EEG) is a method for recording electric potentials stemming from neural activity at the surface of the human head, and it has important scientific and clinical applications. An important issue in EEG signal analysis is the so-called EEG inverse problem where the goal is to localize the source generators, that is, the neural populations that are generating specific EEG signal components. An important example is the localization of the seizure onset zone in EEG recordings from patients with epilepsy. A drawback of EEG signals is however that they tend to be difficult to link to the exact neural activity that is generating the signals.
%
% Source localization from EEG signals has been extensively investigated during the last decades, and a large variety of different methods have been developed. Source localization is very technically challenging: because the number of EEG electrodes is far lower than the number of neural populations that can potentially be contributing to the EEG signal, the problem is mathematically under-constrained. Additional constraints on the number of neural populations and their locations must therefore be introduced to obtain a unique solution.
%
% For the purpose of analyzing EEG signals, the neural sources are treated as equivalent current dipoles. This is because the electric potentials stemming from the neural activity of a population of neurons will tend to look like the potential from a current dipole when recorded at a sufficiently large distance, as in EEG recordings. Source localization is therefore typically considered completed when the locations of the current dipoles have been obtained. Tools for calculating EEG signals from biophysically detailed neural simulations have however recently been developed, and are available through the software LFPy 2.0 (Hagen et al., 2018; Næss et al., 2021). This allows for simulations of different types of neural activity and the resulting EEG signals, opening up for a more thorough investigation of the link between EEG signals and the underlying neural activity.
%
% The past decade has seen a rapid increase in the availability and sophistication of machine learning techniques based on artificial neural networks (ANNs). These methods have also been applied to EEG source localization with promising results. Prior studies have shown ANNs achieve localization errors below 5$\%$, with low computational time and resistance to measurement noise when compared to classical methods (Hoey et al., 2000). Tankelevich (2019) introduced a deep feed-forward network that identified the correct set of source clusters from scalp signals, marking a significant advance in distributed dipole solutions. In a recent study, the authors explored the feasibility of Convolutional Neural Networks (CNNs) to solve the EEG inverse problem. Their CNN, ConvDip, was designed to detect multiple sources using training data adhering to biologically plausible constraints. In this Master’s thesis, the aim will be to investigate the possibility of using both a fully-connected neural network and a convolutional neural network to localize current dipoles and identify strength and radius of clusters of dipoles, based on self-simulated EEG data with tools available through the recently developed software LFPy 2.0 (Hagen et al., 2018; Næss et al., 2021).


% References:
%
% Grech, R., Cassar, T., Muscat, J., Camilleri, K. P., Fabri, S. G., Zervakis, M., ... & Vanrumste, B. (2008). Review on solving the inverse problem in EEG source analysis. Journal of neuroengineering and rehabilitation, 5(1), 25.
%
% Hämäläinen, M., Hari, R., Ilmoniemi, R. J., Knuutila, J., & Lounasmaa, O. V. (1993). Magnetoencephalography—theory, instrumentation, and applications to noninvasive studies of the working human brain. Reviews of modern physics, 65(2), 413.@
\section{Structure of the Thesis}
\end{document}