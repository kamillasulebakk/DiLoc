% !TEX root = main.tex
\documentclass[a4paper, UKenglish, 11pt]{uiomaster}
\usepackage{lipsum}
\usepackage[subpreambles=true]{standalone}

\begin{document}

\chapter{Introduction}

\section{Motivation}

Electroencephalography (EEG) is a method for recording electric potentials stemming from neural activity at the surface of the human head, and it has important scientific and clinical applications. An important issue in EEG signal analysis is the so-called EEG inverse problem where the goal is to localize the source generators, that is, the neural populations that are generating specific EEG signal components. An important example is the localization of the seizure onset zone in EEG recordings from patients with epilepsy. A drawback of EEG signals is however that they tend to be difficult to link to the exact neural activity that is generating the signals.

Source localization from EEG signals has been extensively investigated during the last decades, and a large variety of different methods have been developed. Source localization is very technically challenging: because the number of EEG electrodes is far lower than the number of neural populations that can potentially be contributing to the EEG signal, the problem is mathematically under-constrained. Additional constraints on the number of neural populations and their locations must therefore be introduced to obtain a unique solution.

For the purpose of analyzing EEG signals, the neural sources are treated as equivalent current dipoles. This is because the electric potentials stemming from the neural activity of a population of neurons will tend to look like the potential from a current dipole when recorded at a sufficiently large distance, as in EEG recordings. Source localization is therefore typically considered completed when the locations of the current dipoles have been obtained. Tools for calculating EEG signals from biophysically detailed neural simulations have however recently been developed, and are available through the software LFPy 2.0 \cite{LFPy}. This allows for simulations of different types of neural activity and the resulting EEG signals, opening up for a more thorough investigation of the link between EEG signals and the underlying neural activity.

The past decade has seen a rapid increase in the availability and sophistication of machine learning techniques based on artificial neural networks (ANNs). These methods have also been applied to EEG source localization with promising results. Prior research has demonstrated ANNs achieving localization errors below 5$\%$, exhibiting computational efficiency, and robustness against measurement noise compared to traditional methods \cite{van2000eeg}. In 2019 Tankelevich introduced a deep feed-forward network capable of discerning the correct source clusters within scalp signals, representing a notable advancement in distributed dipole solutions \cite{tankelevich2019inverse}. Moreover, in a recent study, the authors embarked on an exploration of Convolutional Neural Networks (CNNs) to tackle the EEG inverse problem. This network was designed to identify multiple sources using training data adhering to biologically plausible constraints \cite{hecker2021convdip}.

In this master's thesis, the primary objective is to investigate the feasibility of employing both a fully-connected neural network and a convolutional neural network for localizing current dipoles. This investigation extends to identifying the strength and radius of dipole clusters, relying on self-simulated EEG data generated with tools available through the LFPy software.





\section{Structure of the Thesis}

In Chapter \ref{chap:intro_neuro}, we provide an introductory overview of the fundamental structure and functions of neurons. This understanding serves as the cornerstone for our exploration in Chapter \ref{chap:eeg}, where we delve into the principles of the EEG method and its associated inverse problem. We examine how EEG signals can be biologically simulated by modeling neural activity as current dipole moments within head models.
Chapter \ref{chap:eeg_data} focuses on the practical simulation of EEG signals. Here, we employ the Python module LFPy in conjunction with The New York Head Model to create simulated EEG signals.
Chapter \ref{chap:fcnn-approach} introduces a fully connected feed-forward neural network (FCNN). This network is designed to address the EEG inverse problem by mapping simulated EEG signals to their corresponding locations of single current dipoles. We discuss the architecture and hyperparameters employed to achieve this.
Chapter \ref{chap:training_FCNN} provides an in-depth exploration of the training techniques used for the FCNN introduced in Chapter \ref{chap:fcnn-approach}. The results and outcomes of the training process, along with the network's overall performance, are presented in Chapter \ref{chap:simple_dipole_FCNN}.
In Chapter \ref{chap:simple_dipole_CNN}, we present an alternative neural network approach. Here, we utilize a convolutional neural network (CNN) to solve the EEG inverse problem. This chapter elucidates the key concepts of CNNs, data adjustments tailored to the CNN architecture, and the accuracy of predictions following the network's training process.
Chapter \ref{chap:two_dipole_FCNN} extends both neural network architectures to the localization of two current dipoles simultaneously. We present the results from training and display the network's performance in this context.
In Chapter \ref{chap:extended_FCNN}, an enhanced version of the FCNN is introduced. We detail modifications made to the simulation of EEG data, allowing the network to identify additional characteristics of current dipoles beyond their positions. We introduce two new problems for the network: predicting both the location and magnitude of the current source in one case and estimating the center, radius, and signal strength of spherical populations of dipoles in another.
Chapter \ref{chap:discussion}, as the final chapter, involves a comprehensive discussion of the results obtained throughout the thesis. Concluding remarks are provided, and suggestions for potential directions in future research studies are discussed.





\section{Code Availability}

All source code to reproduce the simulation of EEG data and results of the neural network approaches are available on GitHub
\footnote{\url{github.com/kamillasulebakk/DiLoc}}.
The machine learning models are implemented in PyTorch. The code is freely available in the git repository referenced above.
The results have been produced on a personal laptop with no GPU acceleration.



\end{document}
