% !TEX root = main.tex
\documentclass[a4paper, UKenglish, 11pt]{uiomaster}
\usepackage{lipsum}
\usepackage[subpreambles=true]{standalone}
\usepackage{graphicx}


\begin{document}

\chapter{Creating EEG Data}
In preparation for the application of neural networks to address the inverse problem, the acquisition of a substantial and appropriate EEG dataset is essential. This chapter focuses on utilizing the New York Head model in conjunction with the current dipole approximation to construct biophysically realistic EEG data.

\section{Simulation of EEG Signals from single dipoles}
\rednote{Include range of x, y, z values and maybe eeg ?}
The New York Head model is integrated into the Python module LFPy. Within LFPy, we use the \texttt{NYHeadModel} class to calculate EEG signals originating from a current dipole moment $p$. The current dipole moments of the LFPy package are expressed in terms of nA$\mu$m, while the EEG signals derived from the NYHM are recomputed into units of mV. For more information about the LFPy module, we refer the reader to \url{https://lfpy.readthedocs.io/en/latest/readme.html#summary}.

The cortical matrix of the NYHM comprises 74,382 discrete points, each corresponding to a possible location for localizing dipole sources. In the context of simulating EEG measurements, the procedure commences with the random selection of positions from these points to serve as the locations for placing dipoles. Each simulated EEG sample entails a solitary dipole positioned at one of the randomly chosen locations. As the primary objective is to address the inverse problem, maintain uniform magnitudes for the dipole signals, as their variation is not of primary concern. By setting these magnitudes to $10^7$ nA $\mu$m, the resulting EEG measurements span a range of approximately -1 to 1 $\mu$V.

To ensure that the dipole orientations are predominantly aligned with the depth direction of the cortex, a rotation procedure is employed for each dipole moment, orienting it perpendicular to the cerebral cortex. Occasionally, this orientation results in the dipole moment pointing toward an EEG electrode. However, in some cases the dipole moment may be directed back into the cortex, before eventually aligning with an EEG electrode after traversing a greater distance. This phenomenon arises due to the complex folding patterns of the human cortex, where the EEG signal contribution of a dipole moment depends on its position within a sulcus or gyrus \cite{naess2021biophysically}.

The NYHM generates EEG signals as time series data, reflecting the structure of real-world measurements. As a result, the inherent format of EEG data deviates from a one-dimensional representation, instead adopting a matrix configuration of 231 times 1601 dimensions. In this arrangement, 231 values represent measurements from scalp recording electrodes, with 1601 time steps marking the temporal progression. However, as mentioned in the preceding chapter, EEG analysis often centers on specific frequency components within each temporal instance of measurement. This practice effectively reduces the multidimensional EEG data and eliminates less relevant data pints.

In our analysis, we have adopted a one-dimensional data approach to pinpoint sources of neural activity. This discrepancy from conventional practices, which involve extracting diverse frequency spectra and analyzing time series to identify anomalies, offers simplification and computational efficiency. Additionally, this transition is well-suited to our simulated data, which lacks confusing signals from brain activity or unclear background noise, simplifying the data preparation before being fed into a neural network. Rather than conducting a time series analysis with frequency extraction, we focus on data from a static dipole representing a chosen time point. This approach results in a one-dimensional EEG signal that effectively encapsulates insights into the spatial distribution of EEG patterns and their relationship with specific dipole source locations within the cerebral cortex.
\rednote{Add plot of NYHM timeseries data?}

\section{The Effect of Dipole Location and Orientation}
According to Naess et al. (2021) \cite{naess2021biophysically}, EEG signals are not particularly sensitive to minor shifts in the precise location of neural current dipoles. This insensitivity can be explained by the fact that relative to the dimensions of individual neurons and the thickness of the human cortex, EEG electrodes are located far away from cortical neural sources.

The \emph{pearson correlation coefficient} is a useful measure within statistics which liks the relationship between two variables. The correlation coefficient, is the ratio between the covariance of the two variables and the product of the standard deviation. Hence, the coefficient is always a single number between -1 and 1, where -1 describes a perfect negative correlation, and 1 indicates a perfect positive correlation. If the correlation coefficient equals 0, it means that there is no linear relationship between the variables \cite{numpy-docs}.

In our exploration of EEG insensitivity to small differences in dipole locations, we utilize the correlation coefficient as a measure. Figure \ref{fig:neighbour_dipoles} illustrates three EEG signals corresponding to dipoles positioned at neighboring points within the NYHM cortical matrix. The central blue dipole is positioned by the green and red dipoles, situated at the leftmost and rightmost positions within the cortical matrix, respectively. The Euclidean distance between the blue and green dipoles is 0.914 mm, while the blue and red dipoles are separated by 1.926 mm.

The EEG signals of the blue and green dipoles in Figure \ref{fig:neighbour_dipoles} exhibit significant overlap, supported by a high correlation coefficient of 0.966, indicating a strong positive relationship between these measurements. Conversely, the EEG signals of the blue and red dipoles show less resemblance, with a correlation coefficient of 0.695, denoting a somewhat smaller linear relationship. Notably, these differences can be attributed to the rotation in the normal vector of the red dipole and the somewhat larger distance between the blue and red dipoles compared to the blue and green dipoles.

The rotation of the normal vector is also evident for the green dipole but is more pronounced for the red dipole. When selecting neighboring dipoles within the NYHM based on distance, rotations in the normal vectors occur due to constraints during data generation that ensure dipole orientation is perpendicular to the cerebral cortex. Given the complex folding of the cortex, such constraints may lead to rotations in the dipole normal vector when shifting the dipole's location, as demonstrated in this example. Consequently, distinct differences in simulated EEG signals may emerge even for neighboring dipoles within the NYHM.

%Despite the common belief that neurons in the upper cortical layers would dominate the EEG due to their proximity to the electrode compared to neurons in deeper layers, such location differences do not significantly affect the EEG signals. This phenomenon can be explained by the fact that the low conductivity of the skull introduces a spatial low-pass filtering effect, which mitigates the impact of location discrepancies.
% Maybe what is meant here is that we therefore only consider the outer corical surface

\begin{figure}[!htb]
    \centering
    \includegraphics[width=\linewidth]{figures/compare_dipoles.pdf}
    \caption{EEG signals plotted against electrode number for three neighbouring dipoles with normal vectors (-0.60, 0.11, 0.79), (-0.75, -0.15, 0.64), (-0.21, 0.32, 0.93) and positions (-12.04, 34.71, 60.07), (-12.37, 33.92, 59.75), (-10.24, 34.61, 60.75). The correlation coefficient between the blue and green dipole is 0.97, while it is 0.70 for the blue and red dipole. \rednote{Which coordinate system? Where is (0,0,0) located?}}
    \label{fig:neighbour_dipoles}
\end{figure}

Figure \ref{fig:gyrus_and_sulcus_EEG} and \ref{fig:dipole_orientation} are borrowed from work done by Tornjørn Ness and Gaute Einevoll, and furter illustrate the impact of dipole orientation on EEG outcomes. Figure \ref{fig:gyrus_and_sulcus_EEG} represent the EEG signals obtained from two manually selected dipole locations within the New York head model. These dipoles are situated in a gyrus and a sulcus, respectively, and exhibit distinct EEG patterns. In general, the contribution of an individual current dipole to the EEG signal is maximized when the dipole is perpendicularly situated within a gyrus, as depicted in Figure \ref{fig:gyrus_and_sulcus_EEG}B. Contrastingly, when a dipole is placed in a sulcus with a perpendicular orientation, a significant EEG contribution may still be observed, however unlike the dipole in the gyrus, it exhibits a more dipolar pattern, as shown in Figure \ref{fig:gyrus_and_sulcus_EEG}C.


\begin{figure}[!htb]
    \centering
    \includegraphics[width=\linewidth]{figures/gyrus_and_sulcus_EEG.png}
    \caption{A: Two selected dipole locations in the New York head model: one in a gyrus (red) and one in a sulcus (blue). The head model is viewed from the side (x, z-plane). Close to the chosen cross-section plane, EEG electrode locations are marked in light blue. Available dipole locations near the cortical cross-section form an outline of the cortical sheet and are marked in pink. The current dipole moment for all cases was $10^7$ nA$\mu$m. B: Interpolated color plot of EEG signal from the gyrus dipole, viewed from the top (x, y-plane). The plotted EEG signal is scaled, with a maximum value of 1.1 $\mu$V. C: Interpolated color plot of EEG signal from the sulcus dipole. The plotted EEG signal is scaled, with a maximum value of 0.7 $\mu$V. This Figure is borrowed from work done by Torbjørn Ness and Gaute Einevoll \cite{naess2021biophysically}.}
    \label{fig:gyrus_and_sulcus_EEG}
\end{figure}

Further, Figure \ref{fig:dipole_orientation} depicts the EEG signals from identical dipoles positioned in various folding patterns of the cortical surface. These patterns align with the previous observations, as they showcases that the orientation of the current dipole moment significantly influences the EEG outcome. Firstly, Figure \ref{fig:dipole_orientation}A and \ref{fig:dipole_orientation}C provide an expanded illustration of the aforementioned scenarios, incorporating additional dipole moments located in a gyrus and a sulcus, respectively. In Figure \ref{fig:dipole_orientation}B, where a collection of dipoles points randomly upwards and downwards, the EEG signal contribution appears to diminish significantly. Conversely, when the dipoles align in the depth direction of the cortex and are distributed across both gyrus and sulcus, we can expect an EEG contribution in between what we saw from Figure \ref{fig:dipole_orientation}A and \ref{fig:dipole_orientation}B, as depicted in Figure \ref{fig:dipole_orientation}D. Lastly, Figure \ref{fig:dipole_orientation}E demonstrates the minimal EEG contribution observed when the dipoles are divided between two opposing sulci.


\begin{figure}[!htb]
    \centering
    \includegraphics[width=\linewidth]{figures/dipole_orientation.png}
    \caption{Different folding patterns of the cortical surface are represented by white dashed lines. EEG signals are calculated from four identical current dipoles with varying orientations. A: Dipoles aligned in the same direction within a gyrus. B: Dipoles pointing in opposite directions within a gyrus. C: Dipoles aligned in the same direction within a sulcus. D: Dipoles distributed between a gyrus and a sulcus, pointing towards the cortical surface. E: Dipoles divided between opposing sulci, pointing towards the cortical surface.
    Each panel features a dipole moment magnitude of 10 nAm, and the dipoles are positioned at the centers of the arrows in the top row. This Figure is borrowed from work done by Torbjørn Ness and Gaute Einevoll \cite{naess2021biophysically}.}
    \label{fig:dipole_orientation}
\end{figure}

% This is not the case for a simple dipole moment, but might be an issue when giving the populations a radii
% In our analysis, we simplify the scenario by considering one current dipole at a time, which allows us to avoid cancellations of potentials and obtain simpler EEG potentials. While this simplification may not capture the full complexity of neural activity, it provides us with a clearer understanding of the relationship between dipole orientation and EEG signals.



\section{Noise}
Experimental EEG recordings inevitably contain noise, which can interfere with the accurate analysis of brain activity. \emph{Artifacts}, which are signals recorded by EEG but originating from sources other than neuronal communication, pose a particular challenge in real-world data. Some artifacts can mimic genuine epileptiform abnormalities or seizures, underscoring the importance of identifying and distinguishing them from true brain waves \cite{sazgar2019eeg}.

Artifacts can be classified into two categories based on their origin. \emph{Physiological artifacts} arise from the patient's own physiological processes, including ocular activity, muscle activity, cardiac activity, perspiration, and respiration. \emph{Technical artifacts}, on the other hand, originate from external factors such as cable and body movements or electromagnetic interferences \cite{bitbrain}.

Filtering techniques are commonly employed to remove artifacts from EEG recordings prior to analysis. However, in the case of simulated EEG data, the need for artifact removal is eliminated as the data inherently do not contain noise. Simulated EEG data can be considered as pre-filtered and preprocessed, ensuring a high signal-to-noise ratio (SNR) \cite{wiki-snr}. Nevertheless, to ensure the data aligns with real-world scenarios and accounts for other technical considerations which we will come back to later, it is necessary to introduce noise to the data before feeding it into the neural network.

In our approach, we recognize that the introduction of noise to the simulated EEG data is an essential step to enhance the robustness of the trained neural network and ensure its ability to handle real EEG recordings effectively. Since the specific characteristics and quantity of noise have not been the primary focus of our study, we have opted for a straightforward approach. Our final dataset incorporates normally distributed noise with a mean of 0 and a standard deviation equal to 10$\%$ of the standard deviation observed in the simulated EEG recordings. By introducing this noise, we introduce random variations around each data point while preserving the overall normalization properties of the dataset.

\begin{figure}[!htb]
    \centering
    \includegraphics[width=\linewidth]{figures/simple_example.pdf}
    \caption{EEG for a sample containing one single current dipole source at a random position within the celebral cortex. As for all samples within the data set, 10 percent of normally distributed noise has been added to the original signal. The EEG measure is seen from both sides (x-, z-plane and y-, z-plane) and above (the x-, y-plane). EEG electrode locations are presented as filled circels, where the color of the fill represents the magnitue of the measured signal for the given electrode. The position of the current dipole moment is marked with a yellow star.}
    \label{fig:eeg_field_1_dipole_example}
\end{figure}

\subsection{Final Dataset}
The final dataset comprises 70, 000 rows, where each row corresponds to a single sample or fictive patient. Within the dataset, there are 231 columns representing the features, which denote the EEG measurements recorded at each electrode - a configuration directly derived from the NYHM. In practice, the data consists of two separete files holding pairs of EEG data and corresponding target data, where x-, y- and z coordinates of different dipole sources are the answer keys.

Figure \ref{fig:eeg_field_1_dipole_example} presents an example of the input EEG data for a single sample, with 10$\%$ noise added. The EEG result obtained from the specific sample contains a solitary current dipole source positioned randomly within the cerebral cortex. The prominent dipolar pattern in the figure indicates that the dipole is located within a sulcus. In the figure the EEG measure is visualized from multiple perspectives, including the x-z plane, y-z plane, and the x-y plane. The electrode locations are represented by filled circles, with the color of the fill indicating the magnitude of the measured signal at each electrode. The position of the current dipole moment is denoted by a yellow star. As indicated by the colorbar in the figure, the EEG signal for the specific sample ranges from -1 to 1~$\mu$V, which is the range that the simulated EEG data for all samples will fall within.



% Before being feed to the DiLoc network for training, the data is splitted into train, validation and test parts. The train- and validation data are the batches of the data set that the network uses during training. Out of the 70 000 samples in the final dateset, 50000 is set off to the purpose of train and validation data. Out of these 50000 sampes, randomly selected 80 percent of the rows are put into the training set. The remaining 20 percent operates as the validation set, which is useful in order to prevent the network to overfit during training. The test set which contains the final 20 000 samples will be used after the training prosess for the purpose of testing how well the model generalizes to new, unseen data.
%
% Prior to being fed into the DiLoc network for training, the dataset was splittied into distinct segments: the train, validation, and test sets. This partitioning is vital for assessing and optimizing the network's performance. Among the 70 000 samples in the final dataset, 50 000 samples are designated for the train and validation data. To ensure a representative and unbiased allocation, 80 percent of these 50 000 samples are randomly assigned to the training set. This training set serves as the core data that the network utilizes during the training process. The remaining 20 percent of the 50 000 samples form the validation set. This set plays the role in preventing overfitting, the phenomenon where the network becomes excessively attuned to the training data and consequently performs poorly on new data. By independently evaluating the model's performance on the validation set throughout training, we can fine-tune the network's parameters to achieve better generalization to unseen data. Once the network completes its training process, the test set comes into play. Comprising 20 000 samples, the test set serves as the benchmark for assessing the model's ability to generalize and make accurate predictions on new data instances. By adhering to this rigorous train-validation-test data partitioning, we ensure a robust evaluation of the DiLoc model's performance and its capacity to effectively handle real-world scenarios with previously unseen data.



\end{document}