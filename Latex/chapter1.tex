\documentclass[a4paper, UKenglish, 11pt]{uiomaster}
\usepackage{lipsum}
\usepackage[subpreambles=true]{standalone}

\begin{document}
% TODO: Write new background, this is just example
\chapter{Background}
Neurobiology is the study of the nervous system, including the structure, function, and development of neurons and neural circuits. The physics of the neuron is an important component of neurobiology, as it involves understanding the mechanisms by which neurons generate and transmit electrical signals. The basic unit of the nervous system is the neuron, which is capable of producing and transmitting electrical signals, or action potentials, across its membrane. These electrical signals are generated by the flow of charged ions into and out of the neuron, and are essential for communication between neurons and the transmission of information throughout the nervous system.

One technique for studying the electrical activity of the brain is electroencephalography (EEG), which measures the voltage fluctuations resulting from the electrical activity of neurons. EEG is a non-invasive technique that involves placing electrodes on the scalp, and has been used to study a wide range of cognitive and neural processes, including perception, attention, and memory. One of the challenges of interpreting EEG signals is the "inverse problem," which involves determining the location and nature of the underlying sources of electrical activity in the brain.

One approach to solving the inverse problem is source localization, which involves estimating the location and strength of the electrical sources in the brain that are responsible for the measured EEG signals. Source localization is a challenging problem due to the complexity of the brain and the fact that EEG signals are affected by a range of factors, including the conductivity of the scalp and the position and orientation of the electrodes. However, there are a number of techniques and algorithms that have been developed to address these challenges, including dipole modeling, distributed source modeling, and beamforming (Hämäläinen et al., 1993; Grech et al., 2008).

Overall, the physics of the neuron, EEG, and source localization are all important components of neurobiology that have contributed to our understanding of the nervous system and its functioning. By combining knowledge of the physical principles of neural signaling with advanced analytical techniques, researchers are able to gain valuable insights into the underlying neural processes that give rise to behavior and cognition.

% References:
%
% Grech, R., Cassar, T., Muscat, J., Camilleri, K. P., Fabri, S. G., Zervakis, M., ... & Vanrumste, B. (2008). Review on solving the inverse problem in EEG source analysis. Journal of neuroengineering and rehabilitation, 5(1), 25.
%
% Hämäläinen, M., Hari, R., Ilmoniemi, R. J., Knuutila, J., & Lounasmaa, O. V. (1993). Magnetoencephalography—theory, instrumentation, and applications to noninvasive studies of the working human brain. Reviews of modern physics, 65(2), 413.
\section{Introduction to Neuroscience}
% What is a neuron ?
% Layers in the brain ?


\section{Head Models and Multicompartmental Modeling }
% Something more about volume conducters

\section{Currents and Potentials in the Brain}
Ohm's law in volume conductors is a more genral statement than its usual form in electrical circuits. It is a linear relationship between vector current density $J$ and the electric field $E$. The law is then expessed as follows:

\begin{equation}
J = \sigma E,
\label{eq:ohms_law}
\end{equation}

where $\sigma$ is the conductivity of the .... (physical material). (Soruce: Electric Fields of the Brain: The Neurophysics of EEG).


































\section{Electroencephalograpy}
\section{EEG Forward modeling}
\section{The Inverse Problem}
\end{document}