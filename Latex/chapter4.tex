\documentclass[a4paper, UKenglish, 11pt]{uiomaster}
\usepackage{lipsum}
\usepackage[subpreambles=true]{standalone}

\begin{document}

\chapter{Implementations}
In this chapther the models and implementations used for the results of localizations of dipoles in the human cortex will be presented.

\section{New York Head Model}
The New York Head Model is a computer model of the human head used to simulate the electrical activity of the brain. The model is based on anatomical and electrical characteristics of 152 adumlt human brains and can with high precision capture tissue types and complex anatomical structures.

The model is solved for 231 electrode locations 

The New York Head Model is a computer model of the human head used to simulate the electrical activity of the brain. It was created by the Electrical Geodesics Incorporated (EGI) in 2004, and is based on the anatomical and electrical characteristics of the head of a typical adult human.

The model consists of a three-dimensional (3D) representation of the head and brain, with detailed information on the geometry and electrical properties of the different tissues and structures within the head. The model includes the scalp, skull, cerebrospinal fluid, gray matter, and white matter. The electrical properties of each of these tissues, such as conductivity and permittivity, are also included in the model.

The New York Head Model is used in research and clinical applications to understand the electrical activity of the brain and diagnose neurological disorders. It is particularly useful for studying the electrical activity of the brain during various cognitive and motor tasks, as well as during seizures and other abnormal brain activity.

The model is also used in the field of non-invasive brain stimulation, where it helps to guide the placement of electrodes or magnetic coils to target specific areas of the brain for therapeutic or diagnostic purposes. By using the model to predict the electrical activity of the brain, researchers and clinicians can optimize the placement of the electrodes or magnetic coils to achieve the desired effect.

Overall, the New York Head Model is an important tool for understanding the electrical activity of the brain and has numerous applications in research and clinical settings.

\section{The Dataset}

\section{The DiLoc Model}

\section{Feed-Forward Neural Network Approach for localizing single dipole sources}
The FFNN that are trained to solve the inverse problem of ours has an input layer of 231 neurons, corresponding to the M = 231 electrode measures of the potentials. The input layer is followed by three hidden layers with 120, 84 and 16 hidden neurons, respectively. The final output layer holds the predicted x-, y- and z- position of the desired dipole source. For the neurons of the input layers we use the linear activation function ReLu, while for the neurons of the hidden and output layers, we chose the much used hyperbolic tangent activation function.

Cost function

\subsection{Training, testing and evaluation}
In order to make an ANN that generalizes well to new data we split our data into training and testing sets. Randomly selecting 80 percent of the rows in the full dataset, we put this into a separate one and call it our training set. The remaining 20 percent is put into the test set. In practice, the training data set consists of pairs of an input vector with EEG signals and the corresponding output vector, where the answer key is the x-, y- and z coordinate of the dipole source. The neural network is then feed with the training data and produces an estimation of the localization of the dipole. The estimation is found by the network through optimizing the parameters $\beta$ minimizing the cost function, or said in other words, through finding parameters for the function that produces the smallest outcomes, meaning the smallest errors. The result provided by the network is then compared with the target, for each input vector in the training data. Adjustment of parameters...

When the network is fully trained, we have a final model fit on the training data set. Feeding the network with the test data set, we can assess the performance of the network. The predictions of the fully trained network can now be compared to the holdout data's true values to determine the model's accuracy.

In figure \ref{fig:single_dipole_accuracy_FFNN} we have provided the bias-variance trade-off for when using Tanh as activation function. We notice that error of the model is approaching 0 and that the variance between the two curves decreases for an increasing number of epochs.



% Notes:
% The ‘lead field’ or ‘forward model’ used for EEG inverse
% modeling relates a current source in the brain to the electric potentials
% measured on the scalp (Sarvas, 1987; Mosher et al., 1999; Baillet et al.,
% 2001; Vatta et al., 2010; Akalin Acar and Makeig, 2013; Vorwerk et al.,
% 2014).

\end{document}